\chapter{System Design}
\label{chapter:system-design}

This chapter will go through the design of the project software. First, the system requirements will be laid out, followed by the general system design. The system design is simple, as the focus is on improving the performance of rendering.

\section{System Requirements}

\subsection{Functional Requirements}

\begin{itemize}
	\item The software will use fragment shaders to render 3D fractals using sphere tracing.
	\item The software will provide user controls for camera motion.
	\item The software will implement a Signed Distance Field for optimization.
	\item The software will implement a temporal cache for optimization.
	\item The software will be capable of measuring the performance of the rendering process.
	\item The software will be capable of writing these measurements to file in an easily readable format.
	\item The software will be configurable through program arguments.
\end{itemize}

\subsection{Non-Functional Requirements}

\begin{itemize}
	\item The software should be easy to use, with helpful instructions in the README file and printed to the terminal on load.
	\item The software user controls should be intuitive.
	\item The software arguments should be clearly explained.
	\item The software should be stable (it should not crash or have obvious bugs).
	\item The software should have a sensible layout in the code repository.
\end{itemize}

\section{System Design}

The project must include several components. One of these is a renderer, capable of rendering 3D fractals to the screen. This must be configurable through the program's arguments, as the different optimization methods will have different requirements. The next component is the implementation of the two optimization techniques. These will affect the configuration of the renderer as well. Lastly, performance measurements must be taken. These should not affect the rendering process.\newline

With these components, the overall system design is simple. There is no GUI to implement, so no overall design or consideration for the user experience. The main design consideration is in the configuration of the various states of the program. A design for the arguments that the program will take, and how they will affect the program at run time, is considered now.

\subsection{Program Arguments}

The various configurations of the system at run time will be decided via the program's arguments. These will be as follows:\newline

The plan is to render more than one type of fractal, so these will need to be selected via the first argument. The type of fractal affects which signed distance function to use, so is needed to select the shaders to load.\newline

The focus of the project is the improvement of performance, so the various methods will need to be compared to each other. The second input to the program will decide which optimization method to use; in this case, the options are no optimization, the SDF and the temporal cache. This will further affect which shaders will get loaded, as well as some other settings in the renderer.\newline

A real time rendering project would not be very useful if it did not allow animation or movement of some kind. With that in mind, it might be of interest to see how the different optimization methods perform when running animations. The third argument will select whether to animate the scene. This will not affect which shaders are loaded, only which data is passed into them.\newline

The last arguments control the measurement of performance. Argument 4 will control whether performance measurements are taken at all, and argument 5 will contain the name of the file to be written out to. Data will be written to the file specified in a readable format.