\chapter{Introduction}
\label{chapter:introduction}

Fractals are self-similar patterns generated using iterative formulae, and have captured the imaginations of many with their beauty and complexity. Generation of three-dimensional fractals has applications in generative artwork, games and even films. This project will focus on improving the performance of real time rendering of 3D fractals, with the eventual motivation to incorporate them into terrain generation for games.

\section{Project Aim}

The aim of this project is to produce software to render 3D fractals in real time using sphere tracing, and implement two methods of improving the speed of doing so. Success will be measured in terms of program features, measurements of render pass times, and visual evidence of performance differences.\newline

Sphere tracing is a form of ray marching, that relies on calculating the approximate difference to the nearest surface in order to suggest a marching distance in each step.\newline

The first optimization method to be investigated is a Signed Distance Field (referred to as an SDF from this point on), which calculates the approximate distance to the surface offline, and stores it in a grid structure. The second optimization method will be a form of temporal caching, which stores the calculated distance from four of the previous frames, and uses it to speed up calculation in the next.

\section{Objectives}

This project will involve four main stages:

\begin{itemize}
	\item Write a renderer that is capable of sphere tracing three-dimensional fractals.
	\item Implement two optimization methods that attempt to improve the rendering speed.
	\item Implement a performance measuring system to capture performance differences.
	\item Measure the performance differences between the optimized and unoptimized rendering.
\end{itemize}

\section{Deliverables}

The main deliverables of this project are:

\begin{itemize}
	\item A piece of software capable of rendering 3D fractals, with two optimization methods to choose from.
	\item The source code for the software, in the form of a GitHub link (in Appendix \ref{appendix:code-repository}).
	\item The report for the MSc project.
\end{itemize}

\section{Ethical, Legal and Social Issues}

The only ethical or legal issue that this project may encounter is the use of third-party software to implement the project aims. To deal with this, the licensing of each piece of third-party software used will be made clear.