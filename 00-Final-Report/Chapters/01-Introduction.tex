\chapter{Introduction}
\label{chapter:introduction}

This project aims to implement two different methods to improve the speed of rendering three-dimensional fractals (or indeed any shape that can be described with a signed distance function). Both methods rely on using precalculated information to speed up the process of sphere tracing of the scene. One method used will be a three-dimensional signed distance field, which calculates the approximate distance to the surface offline, and stores it in a grid structure. The other method will be a form of temporal caching, which stores the calculated distance from four of the previous frames, and uses it to speed up calculation in the next.\newline

This report will be structured as follows:\newline

This chapter will describe the overall project aim, deliverables and methodology, as well as potential risks and their mitigations.\newline

Chapter 2: Background Research. An overview of the theory behind the project, starting with information about fractals themselves, and ending with research on the optimization methods used.\newline

Chapter 3: Implementation. A description of the application written, along with the methods of measuring success, and testing the program for errors.\newline

Chapter 4: Results. A collection of performance measurements and comparisons, along with explanations of the choices made with regards to representative views.\newline

Chapter 5: Evaluation. A discussion of the results obtained, and explanations of why they turned out as they did.\newline

Chapter 6: Conclusion. A summary of the achievements of the project and an overview of plans for future work.

\section{Project Aim}

The aim of the project is to produce software to render three-dimensional fractals in real time using sphere tracing, and implement two methods of improving the speed of doing so. Success will be measured in terms of program features, and measurements of rendering performance to compare the optimization methods.

\section{Deliverables}

There are two deliverables in the project. One is this report, and the other is access (through a link) to a repository containing the program code.

\section{Methodology and Version Control}

The version control system to use will be Git, through GitHub. This was chosen for its ease of use, and the fact that the repository is stored off-site, making it a very useful backup.\newline

This project will involve three main stages:

\begin{itemize}
	\item Renderer that is capable of sphere tracing three-dimensional fractals.
	\item Two optimization methods that attempt to improve the speed of rendering.
	\item Performance measuring system to capture performance differences between methods.
\end{itemize}

Each stage depends on the last, so a Waterfall methodology will be employed.%A full chart of the stages is shown below.

%<Maybe actually include a chart>

\section{Risks and Mitigation}

The table below summarizes the risks involved with this project, along with their mitigation strategies.\newline

\begin{tabular}{||p{0.45\linewidth}|p{0.45\linewidth}||}
	\hline
	Risk & Mitigation Strategy\\
	\hline\hline
	Run out of time before implementing both optimization methods. & Implement the methods one after the other so that one can be discarded if necessary.\\
	\hline
	Failure to increase performance using one or both methods. & Provide detailed explanations as to why this happened. If there is time, experiment with changes to either method.\\
	\hline
	Technological failure. & Keep code repository up to date in GitHub to avoid loss of progress. Make sure I have access to a backup machine if possible, either borrowing one or using a machine on campus.\\
	\hline
\end{tabular}
