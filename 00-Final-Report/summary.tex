<Concise statement of the problem you intended to solve and main achievements (no more than one A4 page)>

<Nice picture>

Computer generated artwork and 3D geometry is a field of intense interest in computer science. In particular, fractals have captured the imagination of many with their beautiful self-similar patterns and infinite detail. Rendering 3D versions of these fractals in real time using sphere tracing (a form of ray tracing) is the focus of this project. The aim is to speed up the rendering process using precalculated information, saving computation time each frame. Two methods were chosen for this purpose. One of these aims to generate information offline, and make it available to the shaders during rendering. The other aims to save information from previous frames, to be sampled in the current one.

<Spiel about fractals and how nice they are>

<Spiel about sphere tracing>

<Spiel about precalculated values>

<Spiel about project intentions>

<Did it work?>
